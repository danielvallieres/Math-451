\documentclass[reqno]{amsart} 
\usepackage{amssymb,latexsym,amsmath,amscd,graphicx,setspace,amsthm,verbatim}
\usepackage[margin = 3 cm]{geometry}


\theoremstyle{plain}
\newtheorem{theorem}{Theorem}[section]
\newtheorem{proposition}{Proposition}
\newtheorem{corollary}{Corollary}
\newtheorem{lemma}{Lemma}
\newtheorem{conjecture}{Conjecture}
\newtheorem{question}{Question}
\newtheorem{problem}{Problem}
      
\theoremstyle{definition}
\newtheorem{definition}{Definition}

\newenvironment{solution}{\paragraph{\emph{Solution}.}}{\hfill$\square$}


\newenvironment{solution1}{\paragraph{\emph{Solution $1$}.}}{\hfill$\square$}
\newenvironment{solution2}{\paragraph{\emph{Solution $2$}.}}{\hfill$\square$}
\newenvironment{solution3}{\paragraph{\emph{Solution $3$}.}}{\hfill$\square$}

\begin{document} 

\title[Homework 2]{Homework 2}

\date{\today} 
\maketitle 


\begin{problem}
Consider the polynomial
$$P(X) = X^{3} - 3X + 1. $$
Combining your Math 465 and Math 451 skills, can you find its zeros??  (Hint:  You should be able to express them as $2 \cos(\text{ something })$.  At some point, you will have to take the third root of a complex number.  If you are not taking Math 465, you can look it up or come ask me about it.)
\end{problem}
\begin{solution1}

\end{solution1}

\begin{problem}
Let $d$ be a square-free integer, and consider 
$$\mathcal{A}_{d} = \{a+b\sqrt{d}: a,b \in \mathbb{Z} \}. $$
Show that $\mathcal{A}_{d} \le \mathbb{C}$.  (When $d=-1$, $\mathcal{A}_{d}$ is called the ring of Gaussian integers, and we denoted it by $\mathcal{G}$ in class.)
\end{problem}
\begin{proof}

\end{proof}

\begin{problem}
Let $d$ be a square-free integer, and consider
$$\mathcal{Q}_{d} =\{s + t \sqrt{d} : s,t \in \mathbb{Q} \}. $$
Show that $\mathcal{Q}_{d} \le \mathbb{C}$.
\end{problem}
\begin{proof}

\end{proof}

\begin{problem}
Just as in the previous problems, let $d$ be a square-free integer.  Find
$$\mathcal{G}^{\times} \text{ and } \mathcal{A}_{-5}^{\times}. $$
(This one might be a bit tricky...remember that the complex absolute value satisfies $|z_{1} \cdot z_{2}| = |z_{1}|\cdot |z_{2}|$...)
\end{problem}
\begin{proof}

\end{proof}

\begin{problem}
Let $d$ be a square-free integer again.  Now find $\mathcal{Q}_{d}^{\times}$.  (Think hard about the inverse of a non-zero complex number and try to extrapolate from there.)
\end{problem}
\begin{proof}

\end{proof}

\begin{problem}
What is ${\rm char}(\mathcal{G})$?  Explain.
\end{problem}
\begin{solution}

\end{solution}

\begin{problem}
Remember that in class, we defined
$$\Omega  = \Big\{ \begin{pmatrix}a & -b \\ b & a \end{pmatrix} : a,b \in \mathbb{R}\Big\},$$
and we roughly showed that $\Omega \le M_{2}(\mathbb{R})$.  Show that the function
$$\varphi: \mathbb{C} \rightarrow \Omega $$
given by
$$z = a + bi \mapsto \varphi(z) =  \begin{pmatrix}a & -b \\ b & a \end{pmatrix} $$
is an isomorphism of rings.
\end{problem}
\begin{proof}

\end{proof}





\end{document} 



