\documentclass[reqno]{amsart} 
\usepackage{amssymb,latexsym,amsmath,amscd,graphicx,setspace,amsthm,verbatim}
\usepackage[margin = 3 cm]{geometry}


\theoremstyle{plain}
\newtheorem{theorem}{Theorem}[section]
\newtheorem{proposition}{Proposition}
\newtheorem{corollary}{Corollary}
\newtheorem{lemma}{Lemma}
\newtheorem{conjecture}{Conjecture}
\newtheorem{question}{Question}
\newtheorem{problem}{Problem}
      
\theoremstyle{definition}
\newtheorem{definition}{Definition}

\newenvironment{solution}{\paragraph{\emph{Solution}.}}{\hfill$\square$}


\newenvironment{solution1}{\paragraph{\emph{Solution $1$}.}}{\hfill$\square$}
\newenvironment{solution2}{\paragraph{\emph{Solution $2$}.}}{\hfill$\square$}
\newenvironment{solution3}{\paragraph{\emph{Solution $3$}.}}{\hfill$\square$}

\begin{document} 

\title[Homework 5]{Homework 5}

\date{\today} 
\maketitle 


\begin{problem}
In the first take-home exam, you showed how to start with an integral domain $R$, and construct its field of fractions, denoted by ${\rm Frac}(R)$.  An equivalence class $[(a,b)] \in {\rm Frac}(R)$ is usually denoted by
$$\frac{a}{b}. $$
\begin{enumerate}
\item Show that the function $\iota:R \rightarrow {\rm Frac}(R)$ defined via
$$r \mapsto \iota(r) = \frac{r}{1} $$
is an injective ring morphism.
\item Show that given any injective ring morphism $\phi:R \rightarrow F$, where $F$ is a field, there exists a unique ring morphism $\psi:{\rm Frac}(R) \rightarrow F$ such that $\psi \circ \iota = \phi$.  (In other words, ${\rm Frac}(R)$ is the ``smallest'' field containing $R$.  Think about that carefully...) (Hint:  Define $\psi$ via $a/b \mapsto \psi(a/b) = \phi(a)\phi(b)^{-1}$...show $\psi$ is well-defined and has the required properties...)
\end{enumerate}
\end{problem}
\begin{solution}

\end{solution} 

\begin{problem}
Explain carefully why 
$${\rm Frac}(\mathcal{G}) \simeq \mathbb{Q}(i). $$
\end{problem}
\begin{solution}

\end{solution} 

\begin{problem}
Let $R$ be a commutative ring.
\begin{enumerate}
\item Show carefully that $\varphi:\mathbb{Z} \rightarrow R$ defined via $n \mapsto \varphi(n) = n \cdot 1_{R}$ is a ring morphism.
\item Show carefully that 
$${\rm ker}(\varphi) = m\mathbb{Z}, $$
where $m = {\rm char}(R)$.
\item Deduce from Noether's isomorphism theorem that $R$ contains an isomorphic copy of the ring $\mathbb{Z}/{\rm char}(R)\mathbb{Z}$.
\item Explain carefully why the characteristic of an integral domain is either $0$ or a prime number.
\end{enumerate}
\end{problem}
\begin{solution}

\end{solution} 

\begin{problem}
Find all irreducible polynomials of degree two or three in $\mathbb{F}_{3}[T]$.
\end{problem}
\begin{solution}

\end{solution} 

\begin{problem}
Show carefully that
$$\mathbb{Q}(\sqrt{3},\sqrt{5}) = \mathbb{Q}(\sqrt{3} + \sqrt{5}). $$
\end{problem}
\begin{solution}

\end{solution} 

\begin{problem}
A complex number $z \in \mathbb{C}$ is called algebraic over $\mathbb{Q}$ if there exists $P \in \mathbb{Q}[T]$ such that $P \neq 0$, and $P(z) = 0$.  Show that the numbers
$$\sqrt{3 + \sqrt{5}} \text{ and } \sqrt{3} + \sqrt{2}i$$
are algebraic over $\mathbb{Q}$.
\end{problem}
\begin{solution}

\end{solution} 

\begin{problem}
Consider the polynomial 
$$P = T^{3} - 2 \in \mathbb{Q}[T]. $$
\begin{enumerate}
\item Is $P$ irreducible in $\mathbb{Q}[T]$?  Why?
\item Does the field $\mathbb{Q}(\sqrt[3]{2})$ contains {\bfseries all} the roots of $P$??  How many roots of $P$ does $\mathbb{Q}(\sqrt[3]{2})$ contain?
\end{enumerate}
\end{problem}
\begin{solution}

\end{solution} 








\end{document} 



