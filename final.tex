\documentclass[reqno]{amsart} 
\usepackage{amssymb,latexsym,amsmath,amscd,graphicx,setspace,amsthm,verbatim}
\usepackage[margin = 3 cm]{geometry}


\theoremstyle{plain}
\newtheorem{theorem}{Theorem}[section]
\newtheorem{proposition}{Proposition}
\newtheorem{corollary}{Corollary}
\newtheorem{lemma}{Lemma}
\newtheorem{conjecture}{Conjecture}
\newtheorem{question}{Question}
\newtheorem{problem}{Problem}
      
\theoremstyle{definition}
\newtheorem{definition}{Definition}

\newenvironment{solution}{\paragraph{\emph{Solution}.}}{\hfill$\square$}


\newenvironment{solution1}{\paragraph{\emph{Solution $1$}.}}{\hfill$\square$}
\newenvironment{solution2}{\paragraph{\emph{Solution $2$}.}}{\hfill$\square$}
\newenvironment{solution3}{\paragraph{\emph{Solution $3$}.}}{\hfill$\square$}

\begin{document} 

\title[Final exam]{Final exam}

\date{\today} 
\maketitle 

We showed in class that a splitting field for a separable polynomial always gives a Galois extension, so this gives a nice way to construct Galois extensions.  We already pointed out in class that $\mathbb{Q}(\sqrt{2},i)/\mathbb{Q}$ is a Galois extension by counting the number of automorphisms in ${\rm Aut}(\mathbb{Q}(\sqrt{2},i)/\mathbb{Q})$.  The goal of the following problem is to show that $\mathbb{Q}(\sqrt{2},i)$ is also a splitting field for some polynomial.  This gives another proof that $\mathbb{Q}(\sqrt{2},i)/\mathbb{Q}$ is a Galois extension.
\begin{problem}
\hspace{1cm}
\begin{enumerate}
\item Show that $\mathbb{Q}(\sqrt{2},i) = \mathbb{Q}(\sqrt{2} + i)$.
\item Find ${\rm min}_{\mathbb{Q}}(\sqrt{2} + i)$.
\item Show that $\mathbb{Q}(\sqrt{2},i)$ is a splitting field for ${\rm min}_{\mathbb{Q}}(\sqrt{2}+i)$.
\end{enumerate}
\end{problem}
\begin{solution}

\end{solution}


\begin{problem}
We pointed out in class that $\mathbb{Q}(\sqrt[3]{2},\zeta_{3})$ is a splitting field for $T^{3} - 2$, and thus it is a Galois extension of $\mathbb{Q}$.  We also explained that $[\mathbb{Q}(\sqrt[3]{2},\zeta_{3}):\mathbb{Q}]=6$, and therefore it follows that ${\rm Gal}(\mathbb{Q}(\sqrt[3]{2},\zeta_{3})/\mathbb{Q})$ is a finite group with cardinality $6$.
\begin{enumerate}
\item As we pointed out in class, a Galois automorphism is uniquely determined by what it does on the generators $\sqrt[3]{2}$ and $\zeta_{3}$.  For instance, one of the Galois automorphisms is given by
$$r_{1}: \sqrt[3]{2} \mapsto \zeta_{3} \sqrt[3]{2} \text{ and } \zeta_{3} \mapsto \zeta_{3}. $$
List all of the remaining Galois automorphisms using a notation similar to the one for $r_{1}$ above.
\item In the previous question, you should have listed $6$ of them.  Calculate their order and show your work.
\item In the previous question, you should have gotten one automorphism of order one, three of order two and two of order three.  Call the order three ones $r_{1}, r_{2}$ and the order two ones $s_{1}, s_{2}, s_{3}$.  Call the identity simply $1$.  Determine the multiplication table of ${\rm Gal}(\mathbb{Q}(\sqrt[3]{2},\zeta_{3})/\mathbb{Q}) = \{1, r_{1}, r_{2}, s_{1}, s_{2}, s_{3} \}$.
\item Is ${\rm Gal}(\mathbb{Q}(\sqrt[3]{2},\zeta_{3})/\mathbb{Q})$ abelian?  What well-known group of small cardinality is it isomorphic to???
\item Although we haven't proved that yet, we pointed out in class the Galois correspondence for finite Galois extensions.  You are now going to play with this a bit.  First, list all the subgroups of ${\rm Gal}(\mathbb{Q}(\sqrt[3]{2},\zeta_{3})/\mathbb{Q}) = \{1, r_{1}, r_{2}, s_{1}, s_{2}, s_{3} \}$ and compare with what you did on the first problem of Homework one.
\item For each of the subgroup $H$ you found above, find the fixed field $\mathbb{Q}(\sqrt[3]{2},\zeta_{3})^{H}$.  Describe these fixed fields explicitly using generators.
\item How many intermediate extensions of $\mathbb{Q}(\sqrt[3]{2},\zeta_{3})/\mathbb{Q}$ do you have an why??
\end{enumerate}

\end{problem}
\begin{solution}

\end{solution}










\end{document} 



