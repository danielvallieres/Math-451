\documentclass[reqno]{amsart} 
\usepackage{amssymb,latexsym,amsmath,amscd,graphicx,setspace,amsthm,verbatim}
\usepackage[margin = 3 cm]{geometry}


\theoremstyle{plain}
\newtheorem{theorem}{Theorem}[section]
\newtheorem{proposition}{Proposition}
\newtheorem{corollary}{Corollary}
\newtheorem{lemma}{Lemma}
\newtheorem{conjecture}{Conjecture}
\newtheorem{question}{Question}
\newtheorem{problem}{Problem}
      
\theoremstyle{definition}
\newtheorem{definition}{Definition}

\newenvironment{solution}{\paragraph{\emph{Solution}.}}{\hfill$\square$}


\newenvironment{solution1}{\paragraph{\emph{Solution $1$}.}}{\hfill$\square$}
\newenvironment{solution2}{\paragraph{\emph{Solution $2$}.}}{\hfill$\square$}
\newenvironment{solution3}{\paragraph{\emph{Solution $3$}.}}{\hfill$\square$}

\begin{document} 

\title[Homework 3]{Homework 3}

\date{\today} 
\maketitle 


\begin{problem}
Let $S$ be a ring and assume that $R_{1}, R_{2} \le S$.  Show that
$$R_{1} \cap R_{2} \le S. $$
\end{problem}
\begin{solution}

\end{solution}

\begin{problem}
Let $p$ be a rational prime number.  Define
$$\mathbb{Z}_{(p)} = \{ a/b \in \mathbb{Q} : p \nmid b\}. $$
Show that $\mathbb{Z}_{(p)} \le \mathbb{Q}$.
\end{problem}
\begin{solution}

\end{solution}

\begin{problem}
Let $\varphi:R \rightarrow S$ be a ring morphism.  Let $R_{1} \le R$ and let $S_{1} \le S$.  Show that
$$\varphi(R_{1}) \le S \text{ and } \varphi^{-1}(S_{1}) \le R. $$
\end{problem}
\begin{solution}

\end{solution}

\begin{problem}
Let $\varphi:R \rightarrow S$ be a ring morphism.  Let $I \trianglelefteq R$ and let $J \trianglelefteq S$.  
\begin{enumerate}
\item Show that
$$\varphi^{-1}(J) \trianglelefteq R. $$
\item In general, do we have
$$\varphi(I) \trianglelefteq S?? $$
(Hint:  Consider the natural embedding $\mathbb{Z} \hookrightarrow \mathbb{Q}$...)
\item Show that if $\varphi$ is surjective, then
$$\varphi(I) \trianglelefteq S. $$
\end{enumerate}
\end{problem}
\begin{solution}

\end{solution}


\begin{problem}
Let $R$ be a ring and let  $I, J \trianglelefteq R$.  Show that
$$I + J := \{i + j : i \in I \text{ and } j \in J \} \trianglelefteq R. $$
\end{problem}
\begin{solution}

\end{solution}










\end{document} 



